%!TeX program = xelatex
\documentclass[12pt,hyperref,a4paper,UTF8]{ctexart}
\usepackage{RUCReport}
\usepackage{listings}
\usepackage{xcolor}
% 定义可能使用到的颜色
\definecolor{CPPLight}  {HTML} {686868}
\definecolor{CPPSteel}  {HTML} {888888}
\definecolor{CPPDark}   {HTML} {262626}
\definecolor{CPPBlue}   {HTML} {4172A3}
\definecolor{CPPGreen}  {HTML} {487818}
\definecolor{CPPBrown}  {HTML} {A07040}
\definecolor{CPPRed}    {HTML} {AD4D3A}
\definecolor{CPPViolet} {HTML} {7040A0}
\definecolor{CPPGray}  {HTML} {B8B8B8}
\definecolor{keywordcolor}{rgb}{0.8,0.1,0.5}
\definecolor{webgreen}{rgb}{0,.5,0}
\definecolor{bgcolor}{rgb}{0.92,0.92,0.92}

\lstset{
    breaklines = true,                                   % 自动将长的代码行换行排版
    extendedchars=false,                                 % 解决代码跨页时,章节标题,页眉等汉字不显示的问题
    columns=fixed,       
    numbers=left,                                        % 在左侧显示行号
    numberstyle=\small,
    frame=none,                                          % 不显示背景边框
    % backgroundcolor=\color[RGB]{245,245,244},            % 设定背景颜色
    keywordstyle=\color[RGB]{40,40,255},                 % 设定关键字颜色
    numberstyle=\footnotesize\color{darkgray},           % 设定行号格式
    commentstyle=\it\color[RGB]{0,96,96},                % 设置代码注释的格式
    stringstyle=\rmfamily\slshape\color[RGB]{128,0,0},   % 设置字符串格式
    showstringspaces=false,                              % 不显示字符串中的空格
    % frame=leftline,topline,rightline, bottomline         %分别对应只在左侧,上方,右侧,下方有竖线
    frame=shadowbox,                                     % 设置阴影
    rulesepcolor=\color{red!20!green!20!blue!20},        % 阴影颜色
    basewidth=0.6em,
}

\lstdefinestyle{CPP}{
    language=c++,                                        % 设置语言
    morekeywords={alignas,continute,friend,register,true,alignof,decltype,goto,
    reinterpret_cast,try,asm,defult,if,return,typedef,auto,delete,inline,short,
    typeid,bool,do,int,signed,typename,break,double,long,sizeof,union,case,
    dynamic_cast,mutable,static,unsigned,catch,else,namespace,static_assert,using,
    char,enum,new,static_cast,virtual,char16_t,char32_t,explict,noexcept,struct,
    void,export,nullptr,switch,volatile,class,extern,operator,template,wchar_t,
    const,false,private,this,while,constexpr,float,protected,thread_local,
    const_cast,for,public,throw,std},
    emph={map,set,multimap,multiset,unordered_map,unordered_set,
    unordered_multiset,unordered_multimap,vector,string,list,deque,
    array,stack,forwared_list,iostream,memory,shared_ptr,unique_ptr,
    random,bitset,ostream,istream,cout,cin,endl,move,default_random_engine,
    uniform_int_distribution,iterator,algorithm,functional,bing,numeric,},
    emphstyle=\color{CPPViolet}, 
}

\lstdefinestyle{Java}{
    language=[AspectJ]Java,
    keywordstyle=\color{keywordcolor}\bfseries
}

\lstdefinestyle{Python}{
    language=Python,
}

%%-------------------------------正文开始---------------------------%%
\begin{document}

%%-----------------------封面--------------------%%
\cover

%%------------------摘要-------------%%
%\begin{abstract}
%
%在此填写摘要内容
%
%\end{abstract}

\thispagestyle{empty} % 首页不显示页码

%%--------------------------目录页------------------------%%
\newpage
\tableofcontents
\thispagestyle{empty} % 目录不显示页码

%%------------------------正文页从这里开始-------------------%
\newpage
\setcounter{page}{1} % 让页码从正文开始编号

%%可选择这里也放一个标题
%\begin{center}
%    \title{ \Huge \textbf{{标题}}}
%\end{center}

\section{模板说明}
本模板主要适用于一些课程的平时论文以及期末论文,默认页边距为2.5cm,中文宋体,英文Times New Roman,字号为12pt(小四)。

编译方式:\verb|xelatex -> bibtex -> xelatex*2|


默认模板文件由以下四部分组成:
\begin{itemize}
    \item \texttt{main.tex} 主文件
    \item \texttt{reference.bib} 参考文献,使用bibtex
    \item \texttt{RUCReport.sty} 文档格式控制,包括一些基础的设置,如页眉、标题、学院、学号、姓名等
    \item \texttt{figures} 放置图片的文件夹
\end{itemize}

第一次使用时需前往\texttt{RUCReportReport.sty} 对标题、姓名、学号、页眉等进行设置,设置完后即可一劳永逸,封面LOGO亦可替换。

默认带有封面页以及目录页,页码从目录页开始。

\section{一些插入功能}
\subsection{插入公式}
行内公式$v-\varepsilon+\phi=2$。

插入行间公式如\autoref{Euler}:
\begin{equation}
    v-\varepsilon+\phi=2
    \label{Euler}
\end{equation}

\subsection{插入图片}
RUC校徽如\autoref{RUC}所示,注意这里使用了\verb|~\autoref{}|命令,也就是会自动生成“图”“式”等前缀,无需手动输入。

\begin{figure}[!htbp]
    \centering
    \includegraphics[width =.5\textwidth]{figures/ruc_logo.eps}
    \caption{中国人民大学}
    \label{RUC}
\end{figure}

插入上面图片的代码:

\begin{verbatim}
    \begin{figure}[!htbp]
        \centering
        \includegraphics[width =.5\textwidth]{figures/ruc_logo.eps}
        \caption{中国人民大学}
        \label{RUC}
    \end{figure}
\end{verbatim}

\subsection{插入文本框}
本模板定义了一个圆角灰底的文本框,使用简化命令\verb|\tbox{}|即可,如果你不喜欢,可以前往 \texttt{RUCReport.sty}对其进行修改。

\tbox{
    这是一个圆角灰底的文本框
}

\subsection{插入表格}
本模板文件如表~\ref{doc} 所示。
\begin{table}[!htbp]
    \centering
    \begin{tabular}{l  | l}
    \hline
        文件名 & 说明 \\
        \hline
        \texttt{main.tex}  & 主文件 \\
        \texttt{reference.bib} & 参考文献 \\
        \texttt{RUCReport.sty}  & 文档格式控制\\
        \texttt{figures}  & 图片文件夹 \\
        \hline
    \end{tabular}
    \caption{本模板文件组成}
    \label{doc}
\end{table}

%\section{定理环境}
%\begin{Theorem}
%\end{Theorem}
%
%\begin{Lemma}
%\end{Lemma}
%
%\begin{Corollary}
%\end{Corollary}
%
%\begin{Proposition}
%\end{Proposition}
%
%\begin{Definition}
%\end{Definition}
%
%\begin{Example}
%\end{Example}
%
%\begin{proof}
%\end{proof}

\subsection{插入高亮代码块}
利用\verb|lstlisting| 配置
\begin{lstlisting}[style=CPP, title="c++代码"]
#include <iostream>
#include <array>
int main()
{
    constexpr int MAX = 100;
    std::array<int, MAX> arr;
}  
\end{lstlisting}

\begin{lstlisting}[style=Java, title="Java代码"]
public void addAdvertisement(String company, String ad_Category, String ad_Type, String ad_Price)
{
    int price = Integer.parseInt(ad_Price);
    ad = new Advertisement(company, ad_Category, ad_Type, price);
    adList.add(index, ad);
    index++;
    anDM = getDefaultDirectoryManager();
    ActorTuple tuple = new ActorTuple(getActorName(), "advertiser",
    company, ad_Category, ad_Type, price, index-1);
    send(anDM, "register", tuple);
}
\end{lstlisting}

\begin{lstlisting}[style=Python, title="Python代码"]                
import random
import collections
Card = collections.namedtuple('Card', ['rank', 'suit'])

class FrenchDesk:
    ranks = [str(n) for n in range(2, 11)] + list('JQKA')
    suits = 'spades diamonds clubs hearts'.split()
    
    def __init__(self):
        self._cards = [Card(rank, suit) for rank in self.ranks for suit in self.suits]
        
    def __len__(self):
        return len(self._cards)
        
    def __getitem__(self, position):
        return self._cards[position]
desk = FrenchDesk()
\end{lstlisting}

\subsection{插入参考文献}
直接使用\verb|\cite{}|即可。

例如:


   \textit{ 此处引用了文献\cite{0Isaac}。此处引用了文献\cite{2016The}}


引用过的文献会自动出现在参考文献中。

\section{写在最后}
\subsection{发布地址}
\begin{itemize}
    \item Github: \url{https://github.com/xxmy7/RUC_Report_Latex_Template}
    \item Overleaf:  \url{https://www.overleaf.com/latex/template}
\end{itemize}

%%----------- 参考文献 -------------------%%
%在reference.bib文件中填写参考文献,此处自动生成

\reference


\end{document}